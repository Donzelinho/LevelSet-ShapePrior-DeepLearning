\IEEEPARstart{I}{mage} segmentation has been a hot research direction for the past decades which accurately mark the desired regions in the image \cite{Introduction:pal1993review}. And the deep learning has refreshed the research methods of many problems and becomes a watershed of "traditional" methods and "deep" methods \cite{Introduction:lecun2015deep}. Similarly, these methods for the image segmentation are also divided into traditional methods and deep learning methods.

Most of the traditional methods for the image segmentation are based on the global or local statistics information of a single image. And the image segmentation process for the target regions is mainly based on the pre-defined statistical assumptions. Threshold-based methods determined adaptively the threshold grayscale based on the global or local grayscale histogram of images \cite{Introduction:traditional:threshold:sezgin2004survey}. And edge-based methods relied on "the regions in the image uniquely determine the edges". And the edges was detected by the edges detection methods to determine the segmentation regions \cite{Introduction:traditional:edge:senthilkumaran2009edge}. Region-based methods obtained the final segmentation regions by dividing and merging similar regions \cite{Introduction:traditional:watershed:nguyen2003watersnakes, LevelSet:superpixels:achanta2012slic}. Graph-based methods used the graph structure to represent images, and then cut connections of the graph to achieve the image segmentation indirectly \cite{Introduction:traditional:graph:felzenszwalb2004efficient}. The mean shift method mapped all points to the high-dim feature space, and divided the regions based on the mean shift clustering \cite{Introduction:traditional:meanShift:comaniciu2002mean}. Active contour-based methods modeled the target contour explicitly \cite{Introduction:traditional:snakes:kass1988snakes} or implicitly \cite{LevelSet:chan2001active}, and the target contours were obtained by minimizing a energy function. However, these traditional methods are viewed as unsupervised methods, which are based on artificially defined patterns rather than patterns learned from labeled segmentation results.

Different from the traditional methods, deep learning methods are more inclined to find the patterns in images through the training set. Deep neural networks (DNNs) have powerful ability to represent high-level features, so the image segmentation task in DNNs is extended to the semantic segmentation by segmenting regions with the complex high-level semantic information. Deep Learning methods for the image segmentation predict the category of each pixel in images. Fully convolutional networks (FCNs) replace fully connected layers in the network with convolutional layers to accomplish this dense prediction \cite{FCN-original:long2015fully}. However, deep learning methods for the image segmentation have the disadvantages of the noise, boundary roughness and no prior shape \cite{Introduction:FCN:ronneberger2015u, Introduction:FCN:badrinarayanan2017segnet}. Therefore, the improved FCN bsed on Conditional Random Fields (CRFs) is used to solve the problems of noisy and imprecise at boundaries \cite{Introduction:FCN:chen2018deeplab, FCN:CRF:zheng2015conditional}.

Deep convolutional networks (DCNNs) can effectively extract the hidden patterns in images and learn realistic image priors from a large number of example images \cite{Introduction:FCN:prior:UlyanovVL17}. The prior obtained by the deep learning was used to initialize the surface of the level set as the shape prior in the iterative process \cite{Introduction:deep:LevelSet:hu2017deep, Introduction:deep:LevelSet:tang2017deep}. However, the prior knowledge generated by the deep leaning is also rough and imprecise. And the inherent shape prior of the target do not be taken into consideration.

Based on the shape prior representing the intrinsic shape of the target, this paper proposes a level set with deep prior method for the image segmentation based on the priors learned by FCNs. The shape prior is adjusted with the affine transformation to fit a specific image by Global Affine Transformation (GAT). And then the information of the original image, the probability map and the corrected shape prior are combined based on the level set method to obtain the segmentation results. Finally, a series of experiments with Portrait data set \cite{FCN:segmentation:shen2016automatic} are used to verify the effectiveness of the proposed method. The experimental results show that the proposed method can obtain more accurate segmentation result than the traditional FCNs.

The rest of this paper is organized as follows. In Section \ref{sec:Propsed Work}, the proposed method is described, and the original image, the probability map and the corrected shape prior is introduced in detail. The Portrait data set and the experimental results are shown in Section \ref{sec:Experiments and Results}. In section \ref{sec:Discussion}, some discussion of the image segmentation and the deep learning based on the level let method are described. Finally, the conclusion and future work are given in Section \ref{sec:Conclusion and Future Work}.
